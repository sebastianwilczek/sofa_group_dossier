\section{Conclusion}
\label{sec:Conclusion}
\lhead{Conclusion}

At the beginning of the project, there was a lot of information to learn and the team was quiet optimistic to manage the three epics that the client from \textit{Connected.Football} provided. The first eight weeks were mainly spend understanding the language and frameworks that are used and writing all the necessary documents. It was a great experience working with real customers besides studying everything theoretical at University. The customer was a patient customer who did take his time to help and explain basic concepts of the programming language and frameworks. With the help of \textit{Jira} and \textit{GitHub} a real working scenario came up so students could work the way it is properly done. In the project different roles had been assigned so that everybody had different roles and responsibilities to take care of. After the first eight weeks everybody was familiar with the tech stack and their role in the project.
\newline
After having a basic understanding of the technology the group started developing the features. While working on the first epic plenty of unforeseen problems occurred that required a vast majority of time to understand, debug and fix. Due to these issues a lot of valuable time for the development was lost and sprints were overdue. Problems accumulated and the group was stuck at certain points by trying to fix tough errors, e.g. \textit{NPM} updating packages in the background without everybody knowing and breaking the application.
\newline
Roughly a month before the end of the project the group realised that they are far from successfully finishing the project. Every distraction was removed and the main focus was polishing up the results and communicating failure with the customer. The customer was nice and the only thing he wished for was a well documented handover document so that a new group of students can pick up the code and expand it.
\newline
Even though one epic was successfully finished the group learned a lot about \textit{JavaScript}, functional programming, \textit{NPM}, Firebase and all the used frameworks like \textit{React}, \textit{React Native}, \textit{recompose} and \textit{GraphQL}. With this valuable knowledge the group is able to get into \textit{JavaScript} code faster and work themselves through \textit{React Native} applications.